\documentclass[11pt]{article}

\usepackage{amsmath,amssymb,amsthm,setspace,tabto,fancyhdr,sectsty,graphicx}
\usepackage[shortlabels]{enumitem}
\usepackage[nobreak=true]{mdframed}
\usepackage[left=1.25in,right=0.75in,top=1.25in,bottom=2.0in]{geometry}

\newcommand*{\Question}[1]{\section{#1}}
\newenvironment{Parts}{\begin{enumerate}[label=(\alph*)]}{\end{enumerate}}
\newcommand*{\Part}{\item}


%%%%%%%%%%%%%%%%%%%%%%%%%%%%%%%% UNCOMMENT NEXT LINE FOR SOLUTION BOXES
 \newcommand*{\solnboxes}{}

%%%%%%%%%%%%%%%%%%%% name/id
\rfoot{\small Andy Zhang | 3031931319}

%%%%%%%%%%%%%%%%%%%% hw number
\newcommand*{\hwnum}{8}


\ifdefined\solnboxes
    \newenvironment{Answer}{\vspace{10pt}\begin{mdframed}\textbf{Solution}\\}{\end{mdframed}\vfill\pagebreak[3]}
\else
    \newenvironment{Answer}{\vspace{10pt}}{\vfill\pagebreak[3]}
\fi
\newcommand*{\MC}[1]{\multicolumn{1}{c}{#1}}
\newcommand*{\N}{\mathbb{N}}
\newcommand*{\Z}{\mathbb{Z}}
\newcommand*{\Q}{\mathbb{Q}}
\newcommand*{\R}{\mathbb{R}}
\newcommand*{\C}{\mathbb{C}}
\newcommand*{\GF}{\text{GF}}

\pagestyle{fancy}
\headheight=75pt
\sectionfont{\Large\fontfamily{lmdh}\selectfont}

\renewcommand{\headrulewidth}{6pt}
\chead{\rule{\textwidth}{6pt} \vspace{20pt}\\}
\lhead{\setstretch{1.05}\Large\fontfamily{lmdh}\selectfont
CS 70        \tabto{96pt} Discrete Mathematics and Probability Theory\smallskip\\
Spring 2017  \tabto{96pt} Rao}
\rhead{\huge     \fontfamily{lmdh}\selectfont     HW \hwnum}

\lfoot{\small CS 70, Spring 2017, HW \hwnum}
\begin{document}

\Question{Sundry} 
\vspace{10pt}
%%%%%%%%%%%%%%%%%%%% SUNDRY PART HERE
\noindent Nadir Akhtar: nadir\_akhtar@berkeley.edu\\
Rustie Lin: rustielin@berkeley.edu\\
Sukrit Arora: sukrit.arora@berkeley.edu\\

I certify that all solutions are entirely in my words and that I have not looked at another student’s
solutions. I have credited all external sources in this write up. - Andy Zhang
%%%%%%%%%%%%%%%%%%%% END SUNDRY
\vfill\pagebreak[3]

%%%%%%%%%%%%%%%%%%%% QUESTIONS START HERE
\Question{Story Problems}

\newcommand{\sblank}{\vspace{1in}}
Prove the following identities by combinatorial argument:
\begin{Parts}
\Part $\binom{2n}{2} = 2 \binom{n}{2} + n^2$
\begin{Answer}
    To choose 2 objects from $2n$ objects, we can first divide the objects into 2 equally sized sets of $n$ objects. From here, we have the following possibilities: choose both objects from the first set (for which there are $\binom{n}{2}$ ways), choose both objects from the second set (for which there are $\binom{n}{2}$ ways), or choose one object from  either set (for which there are $n^2$ ways). The sum of all three possibilities yields the RHS of the equality.  
\end{Answer}

\Part $n^2 = 2 \binom{n}{2} + n$
\begin{Answer}
    The number of ordered pairs $(i,j)$ where $i$ and $j$ are between 1 and $n$ inclusive is $n^2$. Another way to count this would be to first count all the pairs where $k>j$ (for which there are $\binom{n}{2}$ ways), then all the pairs where $k<j$ (for which there are $\binom{n}{2}$ ways), and finally those where $k=j$ (for which there are $n$ ways). The sum of all three values yields the RHS of the equality.
\end{Answer}

\Part $\sum_{k=0}^n k {n \choose k} = n2^{n-1}$ \\
\textit{Hint:} Consider how many ways there are to pick groups of people ("teams") and then a representative ("team leaders").
\begin{Answer}
    On the LHS, a team of $k$ people is selected from a set of $n$ people. Then, one team leader is selected from this team. On the RHS, the team leader is first selected followed by a selection of a subset of the $n$ people. 
\end{Answer}

\Part $\sum_{k=j}^n {n \choose k} {k \choose j} = 2^{n-j} {n \choose j}$ \\
\textit{Hint:} Consider a generalization of the previous part.
\begin{Answer}
    On the LHS, a team of $k$ people is selected from a set of $n$ people where $k \geq j$. Then, $j$ team leaders are selected from this team. On the RHS, the $j$ team leaders are first selected followed by a selection of a subset of the $n$ people. 
\end{Answer}

\end{Parts}

% Carries over from 10M


\Question{Probability Potpourri}

Prove a brief justification for each part.

\begin{Parts}

\Part For two events $A$ and $B$ in any probability space, show that $\Pr(A \setminus B) \geq \Pr(A) - \Pr(B)$.
\begin{Answer}
    The only times when $\Pr(A \setminus B)=\Pr(A) - \Pr(B)$ are when $B$ is a subset of $A$ or when $\Pr(B)=0$. If neither of these are true, then we have two cases:
    \begin{enumerate}
        \item $A$ and $B$ are disjoint: In this case, $\Pr(A \setminus B)$ simply equals $Pr(A)$ because there is no outcome in $B$ that is also in $A$. Furthermore, since $\Pr(A) > \Pr(A)-\Pr(B)$ for all $\Pr(B)>$, we know that $\Pr(A \setminus B)$ is also greater than $\Pr(A)-\Pr(B)$.
        
        \item $A$ and $B$ are not disjoint: In this case, $\Pr(A \setminus B)=\Pr(A)-\Pr(A \cap B)$. However, since we know that for two non-disjoint events $\Pr(B)>\Pr(A \cap B)$, we arrive at the following inequality: $\Pr(A)-\Pr(A \cap B)>\Pr(A)-\Pr(B)$. Subbing in $\Pr(A \setminus B)$, we get the desired inequality: $\Pr(A \setminus B)>\Pr(A)-\Pr(B)$.
        
    \end{enumerate}
    We have thus shown that in all cases, $\Pr(A \setminus B) \geq \Pr(A) - \Pr(B)$.
\end{Answer}

\Part If $|\Omega| = n$, how many distinct events does the probability space have?
\begin{Answer}
$\sum_{i=0}^{n} \binom{n}{i}$ An event can range in cardinality from 0 to the size of $\Omega$. Thus, we must sum together every possible way to choose any $i$ outcomes from the set of $n$ outcomes to obtain the total number of distinct events of the probability space. 
\end{Answer}

\Part Find some probability space $\Omega$ and three events $A, B$, and $C \subseteq \Omega$ such that $\Pr(A) > \Pr(B)$ and $\Pr(A \mid C) < \Pr(B \mid C)$.
\begin{Answer}
    Suppose we roll a fair six-sided, let $A$ be the event that we get an odd number, $B$ be the event that we roll a $2$, and $C$ be the event that we roll an even number. All conditions will hold because $\Pr(A)=1/2$ which is greater than $\Pr(B)=1/6$. Additionally, $\Pr(A|C)=0$ which is less than $\Pr(B|C)=1/3$.
\end{Answer}

\Part If two events $C$ and $D$ are disjoint and $\Pr(C) > 0$ and $\Pr(D) > 0$, can $C$ and $D$ be independent? If so, provide an example. If not, why not?
\begin{Answer}
    No. For $C$ and $D$ to be disjoint, $\Pr(C \cap D)=0$. Furthermore, for $C$ and $D$ to be independent, $\Pr(C \cap D)=\Pr(C)*\Pr(D)$. For these equalities to both be true, $\Pr(C)*\Pr(D)=0$. However, we know this cannot ever occur because neither $\Pr(C)$ nor $\Pr(D)$ equal 0. Thus, $C$ and $D$ cannot be independent. 
\end{Answer}

\Part Suppose $\Pr(D \mid C) = \Pr(D \mid \overline{C})$, where $\overline{C}$ is the complement of $C$. Prove that $D$ is independent of $C$.
\begin{Answer}
     $\Pr(D \mid C) = \Pr(D \mid \overline{C})$ can be written as $\frac{\Pr(D \cap C)}{\Pr(C)}=\frac{\Pr(D \cap \overline{C})}{\Pr(\overline{C})}$. Since $\Pr(D)$ can be written as $\Pr(D \cap C)+\Pr(D \cap \overline{C})$, we can rewrite the numerator of the RHS as $\Pr(D)-\Pr(D \cap C)$. Additionally we can write the denominator of the RHS as $1-\Pr(C)$. Thus we have the equation: $$\frac{\Pr(D \cap C)}{\Pr(C)}=\frac{\Pr(D)-\Pr(D \cap C)}{1-\Pr(C)}$$ $$\Pr(D \cap C)-\Pr(C)\Pr(D \cap C)=\Pr(C)\Pr(D)-\Pr(C)\Pr(D \cap C)$$ $$\Pr(D \cap C)= \Pr(D)\Pr(C)$$ Thus, since $\Pr(D \cap C)= \Pr(D)\Pr(C)$, we know that $D$ and $C$ are independent.
\end{Answer}

\end{Parts}

\Question{Parking Lots}

Some of the CS 70 staff members founded a start-up company, and you just got hired.
The company has twelve employees (including yourself), each of whom drive a car to work, 
and twelve parking spaces arranged in a row. You may assume that each day all orderings 
of the twelve cars are equally likely.

\begin{Parts}

\Part On any given day, what is the probability that you park next to Professor Rao, 
who is working there for the summer?
\begin{Answer}
    $\frac{1}{6}$
\end{Answer}

\Part What is the probability that there are exactly three cars between yours
and Professor Rao's?
\begin{Answer}
    $\frac{4}{33}$
\end{Answer}

\Part Suppose that, on some given day, you park in a space that is not at one of
the ends of the row.  As you leave your office, you know that exactly five of
your colleagues have left work before you.  Assuming that you remember nothing
about where these colleagues had parked, what is the probability that you will
find both spaces on either side of your car unoccupied?
\begin{Answer}
    $\frac{\binom{9}{3}}{\binom{11}{5}}$
\end{Answer}
\end{Parts}


\Question{Calculate These... or Else is that a threat?}

\begin{Parts}

\Part 
A straight is defined as a 5 card hand such that the card values can be arranged in consecutive ascending order, i.e.\ $\{8,9,10,J,Q\}$ is a straight. Values do not loop around, so $\{Q, K, A, 2, 3\}$ is not a straight. When drawing a 5 card hand, what is the probability of drawing a straight from a standard 52-card deck?
\begin{Answer}
$\frac{9*4^5}{\binom{52}{5}}$
\end{Answer}

\Part
When drawing a 5 card hand, what is the probability of drawing at least one card from each suit?
\begin{Answer}
 $\frac{4*13^3\binom{13}{2}}{\binom{52}{5}}$
\end{Answer}

\Part 
Two squares are chosen at random on $8\times 8$ chessboard. What is the probability that they share a side?
\begin{Answer}
$\frac{112}{\binom{64}{2}}$
\end{Answer}

\Part 
8 rooks are placed randomly on an $8\times 8$ chessboard. What is the probability none of them are attacking each other? (Two rooks attack each other if they are in the same row, or in the same column).
\begin{Answer}
$\frac{8!}{\binom{64}{8}}$
\end{Answer}

\Part A bag has two quarters and a penny. If someone removes a coin, the Coin-Replenisher will come and drop in 1 of the coin that was just removed with $3/4$ probability and with $1/4$ probability drop in 1 of the opposite coin. Someone removes one of the coins at random. The Coin-Replenisher drops in a penny. You randomly take a coin from the bag. What is the probability you take a quarter?
\begin{Answer}
$\frac{8}{15}$
\end{Answer}

\end{Parts}


\Question{Independent Complements}

Let $\Omega$ be a sample space, and let $A,B \subseteq \Omega$ be two independent events.

\begin{Parts}

\Part Prove or disprove: $\overline{A}$ and $\overline{B}$ are necessarily independent.
\begin{Answer}
    We want to show that $\Pr(\overline{A} \cap \overline{B})=\Pr(\overline{A})\Pr(\overline{B})$. By DeMorgan's law, we know that $\overline{A} \cap \overline{B}$ is equivalent to $\overline{A \cup B}$. We thus arrive at the following equation: $$\Pr(\overline{A} \cap \overline{B})=1-\Pr(A \cup B)$$ $$\Pr(\overline{A} \cap \overline{B})=1-\Pr(A)-\Pr(B)+\Pr(A)\Pr(B)$$ $$\Pr(\overline{A} \cap \overline{B})=(1-\Pr(A))(1-\Pr(B))$$ $$\Pr(\overline{A} \cap \overline{B})=\Pr(\overline{A})\Pr(\overline{B})$$
    Since we have shown that $\Pr(\overline{A} \cap \overline{B})=\Pr(\overline{A})\Pr(\overline{B})$, we know that $\overline{A}$ and $\overline{B}$ must be independent.
\end{Answer}

\Part Prove or disprove: $A$ and $\overline{B}$ are necessarily independent.
\begin{Answer}
    Notice that $\Pr(A)$ can be rewritten as $\Pr(A \cap B)+\Pr(A \cap \overline{B})$. We thus have the following equation: $$\Pr(A)=\Pr(A \cap B)+\Pr(A \cap \overline{B})$$ $$\Pr(A)-\Pr(A \cap B)=\Pr(A \cap \overline{B})$$ $$\Pr(A)-\Pr(A)\Pr(B)=\Pr(A \cap \overline{B})$$ $$\Pr(A)(1-\Pr(B))=\Pr(A \cap \overline{B})$$ $$\Pr(A)\Pr(\overline{B})=\Pr(A \cap \overline{B})$$ Since we have shown that $\Pr(A)\Pr(\overline{B})=\Pr(A \cap \overline{B})$, we know that $A$ and $\overline{B}$ must be independent.
\end{Answer} 

\Part Prove or disprove: $A$ and $\overline{A}$ are necessarily independent.
\begin{Answer}
    Suppose $A$ is the event of getting heads after flipping a fair coin. Thus, $\overline{A}$ is the event of getting a tails after flipping a fair coin. $\Pr(A \cap \overline{A})=0$ because you cannot simultaneously get both heads and tails on one flip. However, $\Pr(A)\Pr(\overline{A})=\frac{1}{4}$. Since, $\Pr(A)\Pr(\overline{A}) \not = \Pr(A \cap \overline{A})$, $A$ and $\overline{A}$ are not necessarily independent.
\end{Answer}

\Part Prove or disprove: It is possible that $A=B$.
\begin{Answer}
    Let $A$ and $B$ both be the event of getting heads after flipping a coin that has heads on both sides. Thus, $1=\Pr(A)=\Pr(B)=\Pr(A)\Pr(B)=\Pr(A \cap B)$. Since $\Pr(A)\Pr(B)=\Pr(A \cap B)$, $A$ and $B$ can be both equal and independent.
\end{Answer}

\end{Parts}

\Question{Bag of Coins}

Your friend Forest has a bag of $n$ coins. You know that $k$ are biased with
probability $p$ (i.e.\ these coins have probability $p$ of being heads). Let
$F$ be the event that Forest picks a fair coin, and let $B$ be the event that
Forest picks a biased coin. Forest draws three coins from the bag, but he does
not know which are biased and which are fair.

\begin{Parts}
\Part What is the probability of $FFB$?
\begin{Answer}
$\frac{n-k}{n}(\frac{n-k-1}{n-1})(\frac{k}{n-2})$
\end{Answer}

\Part What is the probability that the third coin he draws is biased?
\begin{Answer}
$\frac{k}{n}$
\end{Answer}

\Part What is the probability of picking at least two fair coins?
\begin{Answer}
$3(\frac{n-k}{n})(\frac{n-k-1}{n-1})(\frac{k}{n-2})+\frac{n-k}{n}(\frac{n-k-1}{n-1})(\frac{n-k-2}{n-2})$
\end{Answer}

\Part Given that Forest flips the second coin and sees heads, what is the
probability that this coin is biased?
\begin{Answer}
$\frac{2pk}{2pk+n-k}$
\end{Answer}
\end{Parts}
%%%%%%%%%%%%%%%%%%%% QUESTIONS END HERE

\end{document}