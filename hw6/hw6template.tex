\documentclass[11pt]{article}

\usepackage{amsmath,amssymb,amsthm,setspace,tabto,fancyhdr,sectsty,graphicx}
\usepackage[shortlabels]{enumitem}
\usepackage[nobreak=true]{mdframed}
\usepackage[left=1.25in,right=0.75in,top=1.25in,bottom=2.0in]{geometry}

\newcommand*{\Question}[1]{\section{#1}}
\newenvironment{Parts}{\begin{enumerate}[label=(\alph*)]}{\end{enumerate}}
\newcommand*{\Part}{\item}


%%%%%%%%%%%%%%%%%%%%%%%%%%%%%%%% UNCOMMENT NEXT LINE FOR SOLUTION BOXES
 \newcommand*{\solnboxes}{}

%%%%%%%%%%%%%%%%%%%% name/id
\rfoot{\small Andy Zhang | 3031931319}

%%%%%%%%%%%%%%%%%%%% hw number
\newcommand*{\hwnum}{6}


\ifdefined\solnboxes
    \newenvironment{Answer}{\vspace{10pt}\begin{mdframed}\textbf{Solution}\\}{\end{mdframed}\vfill\pagebreak[3]}
\else
    \newenvironment{Answer}{\vspace{10pt}}{\vfill\pagebreak[3]}
\fi
\newcommand*{\MC}[1]{\multicolumn{1}{c}{#1}}
\newcommand*{\N}{\mathbb{N}}
\newcommand*{\Z}{\mathbb{Z}}
\newcommand*{\Q}{\mathbb{Q}}
\newcommand*{\R}{\mathbb{R}}
\newcommand*{\C}{\mathbb{C}}
\newcommand*{\GF}{\text{GF}}

\pagestyle{fancy}
\headheight=75pt
\sectionfont{\Large\fontfamily{lmdh}\selectfont}

\renewcommand{\headrulewidth}{6pt}
\chead{\rule{\textwidth}{6pt} \vspace{20pt}\\}
\lhead{\setstretch{1.05}\Large\fontfamily{lmdh}\selectfont
CS 70        \tabto{96pt} Discrete Mathematics and Probability Theory\smallskip\\
Spring 2017  \tabto{96pt} Rao}
\rhead{\huge     \fontfamily{lmdh}\selectfont     HW \hwnum}

\lfoot{\small CS 70, Spring 2017, HW \hwnum}
\begin{document}

\Question{Sundry} 
\vspace{10pt}
%%%%%%%%%%%%%%%%%%%% SUNDRY PART HERE
\noindent Nadir Akhtar: nadir\_akhtar@berkeley.edu\\
Rustie Lin: rustielin@berkeley.edu\\
Sukrit Arora: sukrit.arora@berkeley.edu\\

I certify that all solutions are entirely in my words and that I have not looked at another student’s
solutions. I have credited all external sources in this write up. - Andy Zhang
%%%%%%%%%%%%%%%%%%%% END SUNDRY
\vfill\pagebreak[3]

%%%%%%%%%%%%%%%%%%%% QUESTIONS START HERE
\Question{Error-Correcting Codes}

\begin{Parts}
\renewcommand{\labelenumi}{(\alph{enumi})}
    \Part
    Recall from class the error-correcting code for erasure errors, which
    protects against up to $k$ lost packets by sending a total of $n+k$ packets
    (where $n$ is the number of packets in the original message).  Often the number
    of packets lost is not some fixed number $k$, but rather a \emph{fraction} of
    the number of packets sent.  Suppose we wish to protect against a fraction
    $\alpha$ of lost packets (where $0 < \alpha < 1$).  At least how many packets do 
    we need to send (as a function of $n$ and $\alpha$)?
    \begin{Answer}
        Since $\alpha (n+k)$ packets are lost, the number of packets received must equal $(1-\alpha)(n+k)$. Furthermore since we need $n$ packets to reconstruct the polynomial, $(1-\alpha)(n+k) \geq n$. Solving for $k$, we get $k \geq \frac{\alpha n}{1-\alpha}$. Thus, the minimum value of $k$ is the ceiling of $\frac{\alpha n}{1-\alpha}$. So, the total number of packets sent must be equal to $n+\lceil \frac{\alpha n}{1-\alpha} \rceil$.
    \end{Answer}
    \Part
    Repeat part (a) for the case of general errors.
    \begin{Answer}
        Since at most $\alpha (n+k)$ packets are corrupted and we need 2 additional packets to handle the corruption of a single packet, the number of additional packets $k \leq 2 \alpha (n+k)$. Solving for $k$, we get $k \geq \frac{\alpha n}{1-2\alpha}$. Thus, the minimum value of $k$ is the ceiling of $\frac{\alpha n}{1-2\alpha}$. So, the total number of packets sent must be equal to $n+\lceil \frac{\alpha n}{1-2\alpha} \rceil$.
    \end{Answer}
\end{Parts}

\Question{Polynomials in One Indeterminate}

We will now prove a fundamental result about polynomials: every non-zero
polynomial of degree $n$ (over a field $F$) has at most $n$ roots. If
you don't know what a field is, you can assume in the following that
$F = \mathbb{R}$ (the real numbers).
\begin{Parts}
    \Part Show that for any $\alpha \in F$, there exists some polynomial $Q(x)$
    of degree $n-1$ and some $b \in F$ such that $P(x) = (x-\alpha)Q(x) + b$.
    \begin{Answer}
        Begin by dividing $P(x)$ by $(x-\alpha)$. This will result in a degree $n-1$ polynomial $Q(x)$ summed with a value $\frac{b}{x-\alpha}$ (in the case that there is no remainder, $b=0$). Thus, $P(x)$ can be rewritten as $P(x)=(x-\alpha)(Q(x)+\frac{b}{x-\alpha})=(x-\alpha)Q(x)+b$.
    \end{Answer}

    \Part Show that if $\alpha$ is a root of $P(x)$, then $P(x) =
    (x-\alpha)Q(x)$.
    \begin{Answer}
        In order for $\alpha$ to be a root of $P(x)$, then $P(\alpha)$ must equal 0. Evaluating $P(x)$ at $\alpha$ results in $P(\alpha)=(\alpha-\alpha)Q(\alpha)=0*Q(\alpha)=0$. Thus, $P(x)=(x-\alpha)Q(x)$.
    \end{Answer}

    \Part Prove that any polynomial of degree $1$ has at most one
    root. This is your base case.
    \begin{Answer}
        A polynomial $f(x)$ of degree 1 takes on the form $f(x)=ax+b$. Thus the only solution to $f(x)=0=ax+b$ is when $x=-b/a$. Thus, the proposition holds for the base case.
    \end{Answer}

    \Part Now prove the inductive step: if every polynomial of degree
    $n-1$ has at most $n-1$ roots, then any polynomial of degree $n$ has
    at most $n$ roots.
    \begin{Answer}
        We assume that the for polynomials of degree $n-1$ have at most $n-1$ root. If polynomial $P(x)$ has root \alpha, $P(x)=(x-\alpha)Q(x)$ where $Q(x)$ has degree $n-1$. Since for arbitrary $y$, $P(y)=0 \iff x=y$ or $Q(y)=0$, it follows from the inductive hypothesis that $P(x)$ has at most $n$ roots.
    \end{Answer}
\end{Parts}


\Question{Properties of $\GF(p)$}

\begin{Parts}
    \Part Show that, if $p(x)$ and $q(x)$ are polynomials over the
    reals (or complex, or rationals) and $p(x)\cdot q(x) = 0$ for all
    $x$, then either $p(x)=0$ for all $x$ or $q(x)=0$ for all $x$ or both.
    (\textit{Hint}: You may want to prove first this lemma, true in all fields:
    The roots of $p(x)\cdot q(x)$ is the union of the roots of $p(x)$ and $q(x)$.)
    \begin{Answer}
        Proof by contrapositive: assume that if $p(x) \not = 0$ and $q(x) \not = 0$, then $p(x)*q(x) \not = 0$ for some $x$.In this case, the maximum number of $x$ where $p(x)=0$ is equal to the degree of $p(x)$. In other words, there cannot be infinite solutions to the equation $p(x)=0$. However, since we have an infinite number of $x$ to choose from, there will always exist an $x$ such that $p(x) \not =0$. Because the same argument can be made for $q(x)$, we know there must exist an $x$ such that $p(x)*q(x) \not = 0$.
    \end{Answer}
    
    \Part Show that the claim in part (a) is false for finite fields $\GF(p)$. 
    \begin{Answer}
        Counterexample: choose $p(x)=x$ and $q(x)=x-1$ in GF(2). $p(x)*q(x)=0$ for all $x$, 0 and 1, but neither $p(x)=0$ nor $q(x)=0$.
    \end{Answer}
\end{Parts}


\Question{Poker Mathematics}   

A \emph{pseudo-random number generator} is a way of generating a large quantity of random-looking numbers, if all we have is a little bit of randomness (known as the \emph{seed}). One simple scheme is the \emph{linear congruential generator}, where we pick some modulus $m$, some constants $a,b$, and a seed $x_0$, and then generate the sequence of outputs $x_0,x_1,x_2,x_3,\dots$ according to the following equation:
\[ 
x_{t+1} = ax_t + b \pmod m
\]
(Notice that $0 \le x_t < m$ holds for every $t$.)

You've discovered that a popular web site uses a linear congruential generator to generate poker hands for its players.  For instance, it uses $x_0$ to pseudo-randomly pick the first card to go into your hand, $x_1$ to pseudo-randomly pick the second card to go into your hand, and so on. For extra security, the poker site has kept the parameters $a$ and $b$ secret, but you do know that the modulus is $m=2^{31}-1$ (which is prime).

Suppose that you can observe the values $x_0$, $x_1$, $x_2$, $x_3$, and $x_4$ from the information available to you, and that the values $x_5,\dots,x_9$ will be used to pseudo-randomly pick the cards for the next person's hand. Describe how to efficiently predict the values $x_5,\dots,x_9$, given the values known to you.
\begin{Answer}
    Given the equation for generating pseudo-random numbers, we get the following system of equations: $$x_1 \equiv ax_0+b \mod m$$ $$x_2 \equiv ax_1+b \mod m$$. Since $x_0, x_1, x_2$, we can solve for $a$ and $b$. This is because we know every natural number less than $m$ will have a modular inverse modulo $m$ because $m$ is prime. Thus, with knowledge of $a$ and $b$, we can calculate $x_5, x_6,...,x_9$ using the original formula for generating pseudo-random numbers.
\end{Answer}


\Question{Secret Sharing with Spies}

An officer stored an important letter in her safe. In case she is
killed in battle, she decides to share the password (which is a number)
with her troops. However, everyone knows that there are 3 spies among
the troops, but no one knows who they are except for the three spies
themselves. The 3 spies can coordinate with each other and they will
either lie and make people not able to open the safe, or will open the
safe themselves if they can. Therefore, the officer would like a
scheme to share the password that satisfies the following conditions:
\begin{itemize}
  \item When $M$ of them get together, they are guaranteed to be
          able to open the safe even if they have spies among them.
  \item The 3 spies must not be able to open the safe all by themselves.
\end{itemize}

Please help the officer to design a scheme to share her password. What
is the scheme? What is the smallest $M$? Show your work and argue why
your scheme works and any smaller $M$ couldn't work.
\begin{Answer}
    Choose a polynomial $P(x)$ with degree of at least 3 where $P(0)=s$. If $P(x)$ has degree 3 or less, then the three spies will be able to determine the secret. Thus, $n=4$ because at least 4 points are required to determine a unique degree-3 polynomial. Let the 3 spies symbolize $k=3$ potential corruption errors. Thus a total of $M=n+2k=10$ packets of information must be distributed. Even with the 3 spies, the Berlekamp-Welch algorithm will allow these 10 people to correctly determine $P(x)$. For the Berlekamp-Welch algorithm to work, at least $n+2k$ packets must be sent so $M$ must be at least 10.
\end{Answer}

\Question{Berlekamp-Welch Algorithm}

In this question we will go through an example of error-correcting codes with
general errors.  We will send a message $(m_0,m_1,m_2)$ of length $n = 3$.
We will use an error-correcting code for $k = 1$ general error, doing
arithmetic modulo $5$.

\begin{Parts}
    \renewcommand{\labelenumi}{(\alph{enumi})}
    \Part Suppose $(m_0,m_1,m_2) = (4,3,2)$.  Use Lagrange interpolation to
    construct a polynomial $P(x)$ of degree $2$ (remember all arithmetic is $\bmod
    5$) so that $(P(0),P(1),P(2)) = (m_0,m_1,m_2)$.  Then extend the message to
    length $n+2k$ by appending $P(3),P(4)$.  What is the polynomial $P(x)$ and
    what is the message $(c_0,c_1,c_2,c_3,c_4) = (P(0),P(1),P(2),P(3),P(4))$ that
    is sent?
    \begin{Answer}
        $$\Delta_0=\frac{(x-1)(x-2)}{2}$$
        $$\Delta_1=\frac{(x)(x-2)}{-1}$$
        $$\Delta_2=\frac{(x)(x-1)}{2}$$
        $$P(x)=4 \Delta_0 + 3 \Delta_1 + 2 \Delta_2$$
        $$P(x)=4+4x$$
        $$(c_0,c_1,c_2,c_3,c_4)=(4,3,2,1,0)$$
    \end{Answer}

    \Part Suppose the message is corrupted by changing $c_0$ to $0$.  We will
    locate the error using the Berlekamp-Welch method.  Let $E(x) = x + b_0$ be
    the error-locator polynomial, and $Q(x) = P(x)E(x) = a_3x^3 + a_2x^2 + a_1x +
    a_0$ be a polynomial with unknown coefficients.  Write down the system of
    linear equations (involving unknowns $a_0,a_1,a_2,a_3,b_0$) in the
    Berlekamp-Welch method.  You need not solve the equations.
    \begin{Answer}
        $$a_0=0$$
        $$a_3+a_2+a_1+a_0=3(1+b_0)$$
        $$3a_3+4a_2+2a_1+a_0=2(2+b_0)$$
        $$2a_3+4a_2+3a_1+a_0=1(3+b_0)$$
        $$4a_3+a_2+4a_1+a_0=0$$
    \end{Answer}

    \Part The solution to the equations in part (b) is $b_0 = 0, a_0 = 0, a_1 = 4,
    a_2 = 4, a_3 = 0$.  Show how the recipient can recover the original message
    $(m_0,m_1,m_2)$.
    \begin{Answer}
        The original polynomial $P(x)$ can be recovered by dividing $Q(x)=4x^2+4x$ by $E(x)=x$. Then, evaluating $P(x)$ at $x=0,1,2$ will yield $m_0, m_1, m_2$.
    \end{Answer}
\end{Parts}



\Question{Countability Introduction}

\begin{Parts}
    \Part Do $(0, 1)$ and $\R_+ = (0, \infty)$ have the same cardinality? If so, give an explicit bijection (and prove that it's a bijection). If not, then prove that they have different cardinalities.
    \begin{Answer}
        Let the bijection $f$ be $f:(0, \infty) \rightarrow (0,1)$ and $f(x)=\frac{x}{x+1}$. $f(x)$ is onto because for arbitrary $y\in (0,1)$, $f(\frac{y}{1-y})=y$. To prove $f(x)$ is one-to-one, we will use the contrapositive: Let $x,y \in \R^+$ and $f(x)=f(y)$. Then, $\frac{x}{x+1}=\frac{y}{y+1}$. Simplifying, we get: $$x(y+1)=y(x+1)$$ $$xy+x=xy+y$$ $$x=y$$ Thus, $f(x)$ is also one-to-one. Therefore, $f(x)$ is a bijection.
    \end{Answer}

    \Part Is the set of English strings countable? (Note that the strings may be arbitrarily long, but each string has finite length.) If so, then provide a method for enumerating the strings. If not, then use a diagonalization argument to show that the set is uncountable.
    \begin{Answer}
        Similar to the spiral method used in note 10 to prove $f:\Q \rightarrow \N$ is bijective, we will begin by mapping the subset of $\Z \times \Z$ from 0 to 25 (where 0 is analogous to 'a', 1 is analogous to 'b', etc.) to $\N$ using the spiral method. However, instead of including all points, we will only consider those in the first quadrant. We know that the spiral "function" $s:(0,1,...,25)\times(0,1,...,25) \rightarrow \N$ is a bijection because both the domain and range are subsets of $\Z$ and the note proved that $f:Z \times Z \rightarrow \N$ was a bijection. This mapping will yield all possible combinations of strings of length 2 using the English alphabet. Inductively, we can now repeat this process an arbitrary, finite number of times by crossing the cross-products of the subsets of $\Z$ with another equal subset to include any string of finite length. 
    \end{Answer}

    \Part Consider the previous part, except now the strings are drawn from a countably infinite alphabet $\mathcal{A}$. Does your answer from before change? Make sure to justify your answer.
    \begin{Answer}
        The process described would still work with a countably infinite alphabet because instead of taking a finite subset of $\Z$ to a finite power, we will be taking a countably infinite set (which always has a bijection to $\N$) to the power of a finite number. Since $\N^k$ is countably infinite for any finite $k$, the previous process would still result in countably infinite strings when using a countable infinite alphabet.
    \end{Answer}
\end{Parts}

%%%%%%%%%%%%%%%%%%%% QUESTIONS END HERE

\end{document}