\documentclass[11pt]{article}

\usepackage{amsmath,amssymb,amsthm,setspace,tabto,fancyhdr,sectsty,graphicx}
\usepackage[shortlabels]{enumitem}
\usepackage[nobreak=true]{mdframed}
\usepackage[left=1.25in,right=0.75in,top=1.25in,bottom=2.0in]{geometry}

\newcommand*{\Question}[1]{\section{#1}}
\newenvironment{Parts}{\begin{enumerate}[label=(\alph*)]}{\end{enumerate}}
\newcommand*{\Part}{\item}


%%%%%%%%%%%%%%%%%%%%%%%%%%%%%%%% UNCOMMENT NEXT LINE FOR SOLUTION BOXES
 \newcommand*{\solnboxes}{}

%%%%%%%%%%%%%%%%%%%% name/id
\rfoot{\small Andy Zhang | 3031931319}

%%%%%%%%%%%%%%%%%%%% hw number
\newcommand*{\hwnum}{7}


\ifdefined\solnboxes
    \newenvironment{Answer}{\vspace{10pt}\begin{mdframed}\textbf{Solution}\\}{\end{mdframed}\vfill\pagebreak[3]}
\else
    \newenvironment{Answer}{\vspace{10pt}}{\vfill\pagebreak[3]}
\fi
\newcommand*{\MC}[1]{\multicolumn{1}{c}{#1}}
\newcommand*{\N}{\mathbb{N}}
\newcommand*{\Z}{\mathbb{Z}}
\newcommand*{\Q}{\mathbb{Q}}
\newcommand*{\R}{\mathbb{R}}
\newcommand*{\C}{\mathbb{C}}
\newcommand*{\GF}{\text{GF}}

\pagestyle{fancy}
\headheight=75pt
\sectionfont{\Large\fontfamily{lmdh}\selectfont}

\renewcommand{\headrulewidth}{6pt}
\chead{\rule{\textwidth}{6pt} \vspace{20pt}\\}
\lhead{\setstretch{1.05}\Large\fontfamily{lmdh}\selectfont
CS 70        \tabto{96pt} Discrete Mathematics and Probability Theory\smallskip\\
Spring 2017  \tabto{96pt} Rao}
\rhead{\huge     \fontfamily{lmdh}\selectfont     HW \hwnum}

\lfoot{\small CS 70, Spring 2017, HW \hwnum}
\begin{document}

\Question{Sundry} 
\vspace{10pt}
%%%%%%%%%%%%%%%%%%%% SUNDRY PART HERE
\noindent Nadir Akhtar: nadir\_akhtar@berkeley.edu\\
Rustie Lin: rustielin@berkeley.edu\\
Sukrit Arora: sukrit.arora@berkeley.edu\\

I certify that all solutions are entirely in my words and that I have not looked at another student’s
solutions. I have credited all external sources in this write up. - Andy Zhang
%%%%%%%%%%%%%%%%%%%% END SUNDRY
\vfill\pagebreak[3]

%%%%%%%%%%%%%%%%%%%% QUESTIONS START HERE
\Question{More Countability}

Given:
\begin{itemize}
\item $A$ is a countable set, non-empty set. For all $i \in A$, $S_i$ is an uncountable set.
\item $B$ is an uncountable set. For all $i \in B$, $Q_i$ is a countable set.
\end{itemize}

For each of the following, decide if the expression is
"Always Countable", "Always Uncountable", "Sometimes Countable,
Sometimes Uncountable".

For the "Always" cases, prove your claim. For the "Sometimes" case, provide
two examples -- one where the expression is countable, and one where
the expression is uncountable.

\begin{Parts}
    
\Part $\bigcup_{i \in A} S_i$
\begin{Answer}
Always uncountable. The union of a countable number of uncountable sets will contain at least all elements of the largest of these uncountable sets. Thus, since the smallest possible $\bigcup_{i \in A} S_i$ is still uncountable, no $\bigcup_{i \in A} S_i$ can be countable. 
\end{Answer}

\Part $\bigcap_{i \in A} S_i$
\begin{Answer}
Sometimes countable. When $S_i$ are disjoint, their intersection will be the empty set and will thus be countable. When all $S_i$ are the same, their intersection will be just any one of the $S_i$ which is still uncountable. 
\end{Answer}

\Part $\bigcup_{i \in B} Q_i$
\begin{Answer}
Sometimes countable. If we let $B=\R$ and $Q_i=\N$ for all $i$, the union of all $Q_i$ will be $\N$ whic is countable. If we let $B=\mathbb{R}$ and $Q_i=\{i\}$, then the union of all $Q_i$ will be $\R$ which is uncountable.
\end{Answer}

\Part $\bigcap_{i \in B} Q_i$
\begin{Answer}
Always countable. Since $Q_k$ is countable for any $k\in B$, the union of all $Q_i$ is countable because it is a subset of the set of all $Q_k$. 
\end{Answer}

\Part $A \cap B$
\begin{Answer}
Alwyas Countable. Since $|A \cap B|$ can be at most $|A|$ and we know $A$ is countable, $|A \cap B|$ must also be countable.
\end{Answer}

\end{Parts}


\Question{Counting Cartesian Products}

For two sets $A$ and $B$, define the Cartesian Product as $A \times B = \{(a,b) : a \in A, b \in B \}$.

\begin{Parts}
    \Part Given two countable sets $A$ and $B$, prove that $A \times B$ is countable.
    \begin{Answer}
        In note 10, we proved that $\Z \times \Z$ was countable using the spiral method. This was possible because $\Z$ itself was countable. Similarly, since $A$ and $B$ are both countable, $A \times B$ is also countable. 
    \end{Answer}

    \Part Given a finite number of countable sets $A_1, A_2, \dots, A_n$, prove that 
    $A_1 \times A_2 \times \cdots \times A_n$ \\is countable. 
    \begin{Answer}
        Proceed by induction on $n$.\\
        \\
        Base case: From part (a), we know that $A_1 \times A_2$ is countable\\
        \\
        Inductive hypothesis: For arbitrary $k$, $A_1 \times A_2 \times \cdots \times A_k$ is countable if  $A_1, A_2, \dots, A_k$ are all countable.\\
        \\
        Inductive step: For $k+1$ sets, we know from the inductive hypothesis that $A_1 \times A_2 \times \cdots \times A_{k+1}$ can be rewritten as $A_x \times A_{k+1}$ where $A_x$ is the Cartesian Product of the first $k$ sets. From the inductive hypothesis, we know that $A_x$ is countable. Thus we are left with the Cartesian Product of two countable sets, $A_x$ and $A_{k+1}$ which we know is countable. This completes the induction.
    \end{Answer}

    \Part Consider an infinite number of countable sets: $B_1, B_2, \dots$. Under what
    condition(s) is \\$B_1 \times B_2 \times \cdots$ countable? Prove that if this
    condition is violated, $B_1 \times B_2 \times \cdots$ is uncountable.
    \begin{Answer}
        $B_1 \times B_2 \times \cdots$ is countable when at least one of $B_1, B_2, \dots$ is the empty set or if $B_1, B_2, \dots$ contains an infinite number of cardinality 1 sets and contains a finite number of finite cardinality sets. If these conditions are violated, we can use Cantor's diagonalization to prove that $B_1 \times B_2 \times \cdots=A$ is uncountable. Suppose $A$ is countable, then we can enumerate each element in $A$. For each element, pick two points $x_i, y_i \in B_i$. For every element in each enumeration, if that element equals $x_i$, then change it to $y_i$; otherwise, change it to $x_i$. Using this method, we will obtain an enumeration that is not in $A$ which is a contradiction because we are able to enumerate all elements in $A$. If infinitely many of them have more than one element, their product can be represented as infinitely many infinitely long bit strings which is a bijection to $\R$. Since $\R$ is not countable, $B_1 \times B_2 \times \cdots$ also cannot be countable if the second condition is not met.
    \end{Answer}

\end{Parts}

\Question{Impossible Programs}

Show that none of the following programs can exist.

\begin{Parts}

\Part
Consider a program $P$ that takes in any program $F$, input $x$ and output $y$ and returns true if
$F(x)$ outputs $y$ and returns false otherwise.
\begin{Answer}
     We will construct a counterexample: let $F(x)$ be a program that outputs any value except $y$ if $P(F,x,y)$ returns true and outputs $y$ if $P(F,x,y)$ returns false. Thus, if we pass $F,x,y$ where $x$ and $y$ are arbitrary inputs and outputs to $P$, if $P$ says $F(x)$ will output $y$, then $F(x)$ will not output $y$. If $P$ says $F(x)$ will not output $y$, then $F(x)$ will output $y$. Thus, there exists a program $F$ where $P$ does not return the correct result.   
\end{Answer}

\Part
Consider a program $P$ that takes in any program $F$ and returns true if $F(F)$ halts and returns
false if it doesn't halt.
\begin{Answer}
    We will construct a counterexample: let $F(x)$ be a program that loops forever if $P(F)$ returns true and halts if $P(F)$ returns false. Thus, if we pass $F(F)$ to $P$, if $P$ says $F(F)$ will halt, then it will loop forever. If $P$ says $F(F)$ will loop forever, $F$ will halt. Thus, there exists a program $F$ where $P$ does not return the correct result.
\end{Answer}

\Part
Consider a program $P$ that takes in any programs $F$ and $G$ and returns true if $F$ and $G$ halt
on all the same inputs and returns false otherwise.
\begin{Answer}
     We will construct a counterexample: let $F(x)$ be a program that loops forever for all $x$ if $P(F,G)$ returns true and halts for all $x$ if $P(F,G)$ returns false; let $G(x)$ be a program that halts for all $x$ if $P(F,G)$ returns true and halts if $P(F,G)$ returns false. When passing in $F and G$ to $P$, if $P$ returns true, then $F$ will loop forever for all inputs while $G$ halts for all inputs. If $P$ returns false, then $F$ will halt for all inputs while $G$ will also halt for all inputs. Thus, there exists programs $F$ and $G$ where $P$ will not return the correct result.
\end{Answer}
\end{Parts}


 \Question{Printing All $x$ Where $M(x)$ Halts}

Prove that it is possible to write a program $P$ which:
  \begin{itemize}
  \item takes as input $M$, a Java program,
  \item runs forever, and prints out strings to the console,
  \item for every $x$, if $M(x)$ halts, then $P(M)$ eventually prints out $x$,
  \item for every $x$, if $M(x)$ does NOT halt, then $P(M)$ never prints out $x$.
  \end{itemize}

\begin{Answer}
\begin{verbatim}
    def P(M):
        let i=0
        let lst=[]
        while i<infinity :
            let func_i=a function that runs M on argument i 
            add func_i to list
            run every func in lst for 1 loop
            
            for func_x in list that halted:
                print x
                remove func_x from lst
            i+=1
        
\end{verbatim}
For any $M(x)$ that halts, $x$ will be printed out. Since each function is run only one loop at a time, for any $M(x)$ that does not halt, $x$ will never be printed out and $P$ will continue to iterate over the list of functions $lst$.
\end{Answer}

\Question{Counting, Counting, and More Counting}

The only way to learn counting is to practice, practice, practice, so
here is your chance to do so.
For this problem, you do not need to show work that justifies your answers.
We encourage you to leave your answer as an expression (rather than
trying to evaluate it to get a specific number).
\begin{Parts}
\Part How many 10-bit strings are there that contain exactly 4 ones?
\begin{Answer}
$\binom{10}{4}$
\end{Answer}

\Part How many ways are there to arrange $n$ 1s and $k$ 0s into a sequence?
\begin{Answer}
$\binom{n+k}{k}$
\end{Answer}

\Part A bridge hand is obtained by selecting 13 cards from a standard
  52-card deck. The order of the cards in a bridge hand is
  irrelevant. \\
  How many different 13-card bridge hands are there? 
  How many different 13-card bridge hands are there that contain
  no aces? How many different 13-card bridge hands are there that contain
  all four aces? How many different 13-card bridge hands are there that contain
  exactly 6 spades?
\begin{Answer}
$\binom{52}{13}, \binom{48}{13}, \binom{48}{9}, \binom{13}{5}*\binom{47}{8}$
\end{Answer}

\Part How many 99-bit strings are there that contain more ones than
  zeros?
\begin{Answer}
$\sum_{n=50}^{99} \binom{99}{n}$
\end{Answer}
 
\Part An anagram of FLORIDA is any re-ordering of the letters of FLORIDA, i.e., any
  string made up of the letters F, L, O, R, I, D, and A, in any order.
  The anagram does not have to be an English word. \\
  How many different anagrams of FLORIDA are there? How many different anagrams 
  of ALASKA are there? How many different anagrams of ALABAMA are there? 
  How many different anagrams of MONTANA are there?
\begin{Answer}
$7!, \frac{6!}{3!}, \frac{7!}{4!}, \frac{7!}{2!*2!}$
\end{Answer}

\Part If we have a standard 52-card deck, how many ways are there to
  order these 52 cards?
\begin{Answer}
$52!$
\end{Answer}

\Part Two identical decks of 52 cards are mixed together, yielding a
  stack of 104 cards.
  How many different ways are there to order this stack of 104 cards?
\begin{Answer}
$\frac{104!}{2^{52}}$
\end{Answer}

\Part We have 9 balls, numbered 1 through 9, and 27 bins.
  How many different ways are there to distribute these 9 balls among
  the 27 bins? Assume the bins are distinguishable (e.g., numbered 1
  through 27).
\begin{Answer}
27^9
\end{Answer}

\Part We throw 9 identical balls into 7 bins.
  How many different ways are there to distribute these 9 balls among
  the 7 bins such that no bin is empty? Assume the bins are
  distinguishable (e.g., numbered 1 through 7).
\begin{Answer}
$\binom{8}{2}$
\end{Answer}

\Part How many different ways are there to throw 9 identical balls
  into 27 bins? Assume the bins are distinguishable (e.g., numbered 1
  through 27).
\begin{Answer}
$\binom{35}{9}$
\end{Answer}

\Part There are exactly 20 students currently enrolled in a class.
  How many different ways are there to pair up the 20 students, so
  that each student is paired with one other student?
\begin{Answer}
$\frac{20!}{10!2^{10}}$
\end{Answer}

\Part Let (1, 1) be the bottom-left corner and (8, 8) be the upper-right 
corner of a chessboard. If you are allowed to move one square at a time and
can only move up or right, what is the number of ways to go from the bottom-left corner to 
the upper-right corner? 
\begin{Answer}
$\binom{16}{8}$
\end{Answer}

\Part What is the number of ways to go from the bottom-left corner to 
the upper-right corner of the chesssboard, if you must pass through the square 
(6, 2), where $(i, j)$ represents the square in the $i$th row from the
bottom and the $j$th column from the left?
\begin{Answer}
$\binom{8}{2}*\binom{10}{3}$
\end{Answer}

\Part How many solutions does $x_0 + x_1 + \cdots + x_k = n$ have, if each $x$ must be a non-negative integer?
\begin{Answer}
$\binom{n+k}{k}$
\end{Answer}

\Part How many solutions does $x_0 + x_1 = n$ have, if each $x$ must be a \emph{strictly positive} integer?
\begin{Answer}
$n-1$
\end{Answer}

\Part How many solutions does $x_0 + x_1 + \cdots + x_k = n$ have, if each $x$ must be a \emph{strictly positive} integer?
\begin{Answer}
$\binom{n-1}{k}$
\end{Answer}

\end{Parts}


\Question{Fermat's Necklace}

  Let $p$ be a prime number and let $k$ be a positive integer.  We have an endless supply of beads. The beads come in
  $k$ different colors. All beads of the same color are indistinguishable.

  \begin{Parts}

    \Part We have a piece of string. As a relaxing study break, we want to make a
    pretty garland by threading $p$ beads onto the string.
    How many different ways are there construct such a sequence of $p$ beads of $k$ different colors?
    \begin{Answer}
    $k^p$
    \end{Answer}

    \Part Now let's add a restriction.  We want our garland to be exciting and multicolored. Now
    how many different sequences exist?
    (Your answer should be a simple function of $k$ and $p$.)
    \begin{Answer}
    $k^p-k$
    \end{Answer}

    \Part Now we tie the two ends of the string together, forming a circular
    necklace which lets us freely rotate the beads around the necklace.
    We'll consider two necklaces equivalent if the sequence of colors on one
    can be obtained by rotating the beads on the other.
    (For instance, if we have $k=3$ colors---red (R), green (G), and
    blue (B)---then the length $p = 5$ necklaces RGGBG, GGBGR, GBGRG, BGRGG, and GRGGB are all
    equivalent, because these are cyclic shifts of each other.)

    How many non-equivalent sequences are there now? Again, the $p$
    beads must not all have the same color.
    (Your answer should be a simple function of $k$ and $p$.)

    [\textit{Hint}: What follows if rotating all the beads on a necklace to another
      position produces an identical looking necklace?]
    \begin{Answer}
    $\frac{k^p-k}{p}$
    \end{Answer}

    \Part Use your answer to part (c) to prove Fermat's little theorem.
    (Recall that Fermat's little theorem says that if $p$ is prime and
    $a \not\equiv 0 \pmod p$, then $a^{p-1} \equiv 1 \pmod p$.)
    \begin{Answer}
    Since $\frac{k^p-k}{p}$ must be a natural number, it follows that $p$ divides $k^p-k$. Since $k^p-k=k(k^{p-1}-1)$ and we know that $k$ and $p$ are co-prime, it must be true that $p$ divides $k^{p-1}-1$ which is the statement of Fermat's Little Theorem.
    \end{Answer}

    \end{Parts}

%%%%%%%%%%%%%%%%%%%% QUESTIONS END HERE

\end{document}